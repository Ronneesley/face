\chapter{Introdução}

A rede social é um trabalho desenvolvido pelos alunos do curso de Bacharelado em Sistemas de Informação sob orientação do professor Ronneesley Moura Teles.
O trabalho tem cunho didático e está voltado para o aprendizado de programação WEB e programação Orientada a Objetos.
Este documento descreve as informações básicas sobre a rede social, como seus requisitos e casos de uso.


\section{Escopo do produto}
Rede social é um sistema comunicativo que promove a interatividade entre meios públicos
visando como principal objetivo, despertar a curiosidade e sagacidade das pessoas em relação a
conectividade e o mundo ao seu redor. A rede social concede ao indivíduo a criação de um perfil e a
partir deste, as pessoas que tomam a iniciativa de realizar o ato, são chamadas de usuários.

Os usuários desfrutam da liberdade de desenvolver, modificar e excluir conteúdos tomados
pelo mesmo como post (postar). Assim, foi idealizado com Ronneesley Moura Teles, professor do
Campus Ceres do IF Goiano, a criação de uma mídia social através dos alunos do 4º período de
Bacharelado em Sistemas de Informação. O projeto está em processo de desenvolvimento, porém, a
mídia social não possui um nome definido até o momento, sendo chamada apenas de “rede social”.

O ambiente de interação busca englobar pessoas e fazer com que não apenas docentes e
discentes possam desfrutar da rede social, mas também, aqueles que tiveram algum vínculo com a
instituição (egressos, desistentes) e aos que possuem interesse no mesmo (candidatos,
pesquisadores, palestrantes, entre outros). A rede social oferecerá vários benefícios, sendo eles: a
otimização da interação entre professores, servidores, alunos de ambos os cursos ofertados pelo
Campus Ceres do IF Goiano e zona externa.

Logo, pensando em meios acadêmicos, a rede social poderá ajudar na divulgação de trabalhos
e eventos científicos de forma a contribuir e sempre agregar valor aos estudos. Também deve oferecer postagens de vídeos, fotos, grupos de assuntos acadêmicos, artigos
científicos, dissertações e teses. Além disso, a rede social desenvolvida ocasionará formas positivas de
abrir espaços para discussões sobre os assuntos postados. Entretanto, o projeto realizado não tem a
intenção de servir como um banco de teses/dissertações, mas sim, promover o conhecimento
científico e permitir que pessoas possam tirar dúvidas através dos posts e comentários.
  

\section{Estrutura do documento}

O Capítulo 2 apresenta a perspectiva, funções, características, restrições e suposições e suas pendências do produto tendo como base descrições menos detalhadas do produto. Em seguida, o Capítulo 3 apresenta os requisitos funcionais e não-funcionais, descrevendo o que o \textit{software} realiza e como realiza, desempenho, interface entre outros. Analogamente, o Capítulo 4 refere-se aos protótipos, que são uma visão inicial de um sistema de \textit{software}, possibilitando demonstrar conceitos, experimentar opções de projeto, mostrando toda a interface inicial que talvez sofra alterações futuras. Por fim o Capítulo 5 descreve como será o uso de uma funcionalidade de um sistema, demonstrando o fluxo correto para seu funcionamento, como funcionalidades de alguns botões e níveis de acesso do usuário.