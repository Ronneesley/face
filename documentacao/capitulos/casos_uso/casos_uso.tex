\chapter{Casos de Uso}

\casoDeUso
%Identificador
{UC1}
%Nome
{Realizar cadastro}
%Ator Principal
{Usuário}
%Interessados
{
\begin{itemize}
	\item Usuário: deseja cadastrar-se na rede social.	
\end{itemize}

}
%Pré-Condições
{Ter um e-mail válido.}
%Pós-condições
{Dados são enviados para o servidor.}
%Fluxo Básico
{
\begin{itemize}
	\item Usuário preenche os campos de nome, sobrenome, e-mail e data de nascimento como mostra a Figura  \ref{figura:login}; %login	
	\item Usuário aceita os termos de uso da rede social;		
	\item Usuário clica no botão \textit{“Criar conta”};	
	\item O sistema envia e-mail para o usuário para confirmação como mostra a Figura \ref{figura:cadastro_usuario_parte2};
	\item Após confirmar o e-mail, o usuário é redirecionado para a segunda etapa do cadastro;		
	\item O usuário preenche sua senha;
	\item O usuário preenche informações sobre local de trabalho, instituição de ensino, cidade natal, cidade atual, relacionamento e outras redes sociais como mostra a Figura \ref{figura:cadastro_usuario_parte3};
	\item Ao clicar no botão \textit{“Salvar e continuar”}, o usuário finaliza a segunda etapa de cadastro;	
	\item Usuário tem a opção de anexar uma foto para seu perfil usando sua \textit{webcam} ou faz \textit{upload} de uma foto para o sistema como mostra a Figura \ref{figura:cadastro_usuario_parte3}; %falta a foto do protótipo da 3a etapa
	\item Ao clicar no botão \textit{“Salvar e continuar”}, o usuário finaliza a terceira etapa de cadastro;
	\item O usuário tem acesso a seu perfil.	
	
			
\end{itemize}
}
%Fluxo Alternativo
{
\begin{itemize}
	\item Usuário não preenche o campos de nome, sobrenome, celular e/ou e-mail, e precisa ser notificado;
	\item Usuário não aceita os termos de uso da rede social;
	\item Eventuais problemas no servidor podem impedir o envio de dados, sendo preciso notificar o usuário sobre;
	\item Usuário pode pedir o reenvio do e-mail de confirmação;
	\item Usuário pode não escolher uma foto para seu perfil.
	
\end{itemize}
}
%Frequência de Ocorrencia
{Aproximadamente 3 vezes por minuto.}
%Problemas em Aberto
{

}

%-----------------------------------------------------------

\casoDeUso
%Identificador
{UC2}
%Nome
{Enviar confirmação de cadastro}
%Ator Principal
{Usuário}
%Interessados
{
\begin{itemize}
	\item Usuário: deseja receber o e-mail de confirmação.
\end{itemize}

}
%Pré-Condições
{Ter recebido todos os dados de forma válida}
%Pós-condições
{E-mail de confirmação é enviado}
%Fluxo Básico
{
\begin{itemize}
	\item E-mail é enviado com sucesso pelo servidor;
	\item E-mail é recebido com sucesso pelo usuário.	
\end{itemize}
}
%Fluxo Alternativo
{
\begin{itemize}
\item E-mail é enviado para a caixa de \textit{spam} do usuário.
\end{itemize}
}
%Frequência de Ocorrencia
{Aproximadamente 3 vezes por  minuto.}
%Problemas em Aberto
{
 
}

%-----------------------------------------------------------

\casoDeUso
%Identificador
{UC3}
%Nome
{Confirmar recebimento de confirmação e ativar conta.}
%Ator Principal
{Usuário}
%Interessados
{
\begin{itemize}
	\item Usuário: deseja confirmar e-mail e ganhar acesso a seu perfil na rede social.
\end{itemize}

}
%Pré-Condições
{E-mail de confirmação precisa ter sido enviado para o usuário}
%Pós-condições
{A conta do usuário é ativa}
%Fluxo Básico
{
\begin{itemize}
	\item Usuário clica no link de confirmação;
	\item Usuário é redirecionado para continuação do cadastro prosseguindo para a segunda etapa do cadastro;	
	\item O usuário preenche sua senha;
	\item O usuário anexa informações sobre local de trabalho, instituição de ensino, cidade natal, cidade atual, relacionamento e outras redes sociais como mostra a Figura \ref{figura:cadastro_usuario_parte2};
	\item Ao clicar no botão \textit{“Salvar e continuar”}, o usuário finaliza a segunda etapa de cadastro;
	\item Usuário tem a opção de anexar uma foto para seu perfil usando sua \textit{webcam} ou upando uma foto para o sistema;
	\item Ao clicar no botão \textit{“Salvar e continuar”}, o usuário finaliza a terceira etapa de cadastro; 		
	\item Conta é liberada para o uso do usuário.		 
\end{itemize}
}
%Fluxo Alternativo
{
\begin{itemize}
	\item A confirmação não está funcionando corretamente ou expirou;
	\item Usuário solicita o reenvio da confirmação;
	
	\item Usuário deixa em branco itens opcionais de cadastro;
	\item Usuário pode clicar no botão \textit{“Pular”} ao invés de \textit{“Salvar e continuar”};
	
	\item Sistema informa que usuário não preencheu tais itens opcionais e que eles podem ser posteriormente editados.
	
\end{itemize}
}
%Frequência de Ocorrencia
{Aproximadamente 3 vezes por minuto.}
%Problemas em Aberto
{

}

\casoDeUso
%Identificador
{UC4}
%Nome
{Enviar Post}
%Ator Principal
{Usuário}
%Interessados
{
\begin{itemize}
	\item Usuário: deseja realizar publicações em sua rede social.
\end{itemize}

}
%Pré-Condições
{Ter uma conta válida e ativa}
%Pós-condições
{Públicação é postada}
%Fluxo Básico
{
\begin{itemize}
\item Usuário acessa a sua \textit{timeline};
\item Usuário poderá postar foto, vídeo, \textit{link} ou texto;
\item Apos inserir todas as informações o usuário aciona a opção de publicar \textit{post}.
\end{itemize}
}
%Fluxo Alternativo
{
\begin{itemize}
\item Ocorreu algum erro ao publicar \textit{post}.
\item É mostrado uma mensagem de erro ao inserir.
\item Usuário aciona a opção “Tentar Novamente”.
\end{itemize}
}
%Frequência de Ocorrencia
{Aproximadamente 50 vezes ao dia}
%Problemas em Aberto
{

}

%-----------------------------------------------------------
\casoDeUso
%Identificador
{UC5}
%Nome
{Realizar Login}
%Ator Principal
{Usuário}
%Interessados
{
\begin{itemize}
	\item Usuário: deseja realizar login em sua conta;
	\item Servidor: deseja receber os dados do usuário para validar e realizar o login.
\end{itemize}

}
%Pré-Condições
{Possuir uma conta válida.}
%Pós-condições
{Login é realizado com sucesso.}
%Fluxo Básico
{
\begin{itemize}
\item Usuário acessa o site da rede social;
\item O sistema exibe a página de login (Figura \ref{figura:login});
\item Usuário digita seu usuário e senha, em seus respectivos campos;
\item Após inserir as informações, o usuário aciona a opção de realizar login;
\item Servidor valida as informações digitadas pelo usuário e abre uma sessão para o mesmo.
\end{itemize}
}
%Fluxo Alternativo
{
\begin{itemize}
\item Ocorreu algum erro ao realizar login;
\begin{itemize}
\item É mostrado uma mensagem de erro ao tentar realizar login.
\end{itemize}
\item Usuário não preenche os campos necessários;
\begin{itemize}
\item Usuário não preencheu o campo de e-mail;
\item Usuário não preencheu o campo de senha.
\end{itemize}
\item Usuário digita e-mail ou senha incorretamente.
\begin{itemize}
\item Sistema informa usuário sobre o erro.
\end{itemize}
\item Problemas com o servidor que precisam ser notificados ao usuário;
\begin{itemize}
\item Problemas com a conexão com o banco de dados;
\item Problemas com a conexão com o servidor.
\end{itemize}
\item Problemas com sua conexão com a Internet.
\begin{itemize}
\item É notificado ao usuário falha com a conexão com a Internet.
\end{itemize}
\end{itemize}
}
%Frequência de Ocorrencia
{Aproximadamente 20 vezes ao dia.}
%Problemas em Aberto
{

}


%-----------------------------------------------------------
\casoDeUso
%Identificador
{UC6}
%Nome
{Realizar comentarios}
%Ator Principal
{Usuário}
%Interessados
{
\begin{itemize}
	\item Usuário: Deseja comentar uma postagem;
	\item Autor da Postagem: Deseja ler os comentários publicados.
\end{itemize}

}
%Pré-Condições
{Estar validado no sistema}
%Pós-condições
{Cometário Publicado}
%Fluxo Básico
{
\begin{itemize}
	\item Usuário identifica uma postagem do seu interesse;
	\item Usuário digita o comentário conforme a Figura \ref{figura:meus_posts};
	\item O usuário clica em ``enviar" conforme a Figura \ref{figura:meus_posts};
	\item Comentário é publicado.
\end{itemize}
}
%Fluxo Alternativo
{
\begin{itemize}
	\item Postagem foi deletada;
	\begin{itemize}
		\item Exibe uma mensagem informando que a postagem foi deletada;
		\item Volta para a tela anterior.
	\end{itemize}
	
	\item Falha na conexão com o servidor;
	
	\begin{itemize}
		\item Exibe uma mensagem informando que não foi possível conectar ao servidor;
		\item Volta para a tela anterior.
	\end{itemize}
	
	\item Conexão com a Internet falhou.
	\begin{itemize}
		\item Exibe uma mensagem para verificar a conexão com a Internet;
		\item Volta para a tela anterior.
	\end{itemize}

	
	
\end{itemize}
}
%Frequência de Ocorrencia
{Aproximadamente 10 vezes ao dia.}
%Problemas em Aberto
{

}

%-----------------------------------------------------------
\casoDeUso
%Identificador
{UC7}
%Nome
{Sair da rede social}
%Ator Principal
{Usuário}
%Interessados
{
\begin{itemize}
	\item Usuário: Deseja sair e apagar todas os cachês e \textit{cookies} da rede social.
\end{itemize}
}
%Pré-Condições
{Possuir uma conta válida e estar conectado (logado no sistema)}
%Pós-condições
{Usuário Deslogado}
%Fluxo Básico
{
\begin{itemize}
	\item Clicando no botão  \textit{pop-up} no canto superior direito da tela próximo a foto. Em seguida, basta escolher a opção “Sair”.  como mostra a Figura  \ref{figura:meus_posts}.
\end{itemize}
}
%Fluxo Alternativo
{
\begin{itemize}
	\item Usuário desconectado;
		\begin{itemize}
		\item Exibir mensagem  ao usuário falando que ele está deslogado.
		\end{itemize}
	\item Problemas de resposta do servidor;
		\begin{itemize}
		\item Exibir mensagem  ao usuário relatando problemas com resposta do servidor.
		\end{itemize}
	\item Exibe uma mensagem informando que o usuário foi desconectado;
	\item Volta para a tela inicial do sistema de fazer login.
		\begin{itemize}
		\item Problemas com sua conexão com a internet.
			\begin{itemize}
			\item Exibe uma mensagem informando que o usuário não está conectado a Internet.
			\end{itemize}
		\end{itemize}
	
\end{itemize}
}
%Frequência de Ocorrencia
{Toda vez que o usuário deseja sair da rede social.}
%Problemas em Aberto
{

}
%-------------------------------------------------------------------------
\casoDeUso
%Identificador
{UC8}
%Nome
{Alterar dados de Perfil}
%Ator Principal
{Usuário}
%Interessados
{
\begin{itemize}
	\item Usuário: Deseja alterar seus dados de perfil.	
\end{itemize}
}
%Pré-Condições
{Estar conectado ao sistema e estar na página Perfil.}
%Pós-Condições
{Após alterações realizadas com sucesso, estes dados ser enviados ao servidor.}
%Fluxo Básico
{
\begin{itemize}
	\item Usuário acessa o perfil como mostra a Figura \ref{figura:perfil};
	\item Sistema demonstra o seu perfil;
	\item Usuário clica no menu \textit{``Sobre"};
 	\item Sistema exibe os dados detalhados do perfil do usuário;
	\item Usuário seleciona os campos que deseja alterar clicando nos ícones como mostra a Figura \ref{figura:cadastro_usuario_parte3};
	\item Usuário preenche os campos;
	\item Usuário confirma os campos ao clicar no botão \textit{``Confirmar"};
	\item Sistema armazena os dados inseridos, alterados ou excluídos do usuário;
	\item Sistema detectar algum erro durante o processo, deve demonstrar ao usuário o local.
\end{itemize}
}
%Fluxo Alternativo
{
\begin{itemize}
	\item Sistema \textit{Offline};
	   \begin{itemize}	
	   	\item Exibe uma mensagem informando que não foi possível conectar ao servidor ao usuário;
		\item Volta para a tela inicial do sistema;
		\item Usuário desconectado;
		\item Sistema exibe mensagem  ao usuário falando que ele está deslogado;
		\item Problemas com resposta do servidor ao alterar o campo especificado;
		\item Exibe a mensagem de erro ao usuário;
		\item Usuário modifica os dados, porém, no momento que realizou as alterações, estava sem conexão a internet;
		\item Perda das alterações realizadas, tendo que refazer ao restabelecer a conexão com a Internet, caso tenha efetuado alguma alteração mantêm como estava anteriormente;
		\item Sistema informa ao usuário sobre o devido erro de conexão a Internet.
     	    \end{itemize}
	\item Preenchimento dos campos;    
	    \begin{itemize}
		\item Usuário preenche o campo incorretamente ex:(Caracteres especiais, letras em campos de números, etc).
		\item Exibe mensagem de erro ao usuário sobre o preenchimento do campo;
		\item Exibe a forma correta de preencher;
		\item Usuário insere redes sociais inválidas;
		\item Sistema informa sobre os devidos erros de inserção;
		\item Usuário insere telefone inválido;
		\item Sistema informa ao usuário sobre o telefone inválido;
		\item Sistema demonstra forma correta de inserir o telefone.
	     \end{itemize}
	\item Alteração e confirmação dos campos;     
		\item Usuário não confirma a alteração;
		\item Sistema informa o usuário se deseja salvar as alterações realizadas, se sim guarda os dados, senão mantêm os dados anteriores;
		\item Usuário encerra a sessão sem salvar as alterações;
		\item Sistema não guarda os dados alterados;
		\item Usuário fecha o navegador.
\end{itemize}
}
%Frequência de Ocorrência
{Aproximadamente 3 vezes ao dia}
%Problemas em aberto
{

}
%--------------------------------------------------------------------------------------------------------
\casoDeUso
%Identificador
{UC9}
%Nome
{Pesquisar pessoas e grupos}
%Ator Principal
{Usuário}
%Interessados
{
\begin{itemize}
	\item Usuário: Deseja pesquisar uma pessoa ou grupo.	
\end{itemize}
}
%Pré-Condições
{Estar conectado ao sistema e estar na página Perfil.}
%Pós-Condições
{Após a pesquisa realizada com sucesso, as pessoas/grupos com o nome relacionado ao que foi pesquisado aparecem para que o usúario escolha.}
%Fluxo Básico
{
\begin{itemize}
	\item Usuário: Acessa o seu perfil como mostra a figura 4;
	\item Usuário: Clica no campo de pesquisa;
	\item Usuário: Escrever o nome da pessoa/grupo que deseja pesquisar;
	\item Usuário: Selecionar a pessoa/grupo que aparecem com o nome relacionado ao nome que foi pesquisado;
	\item Usuário: O usúario e redirecionado para a página de perfil da pessoa ou para a página do grupo que foi selecionado.
\end{itemize}
}
%Fluxo Alternativo
{
\begin{itemize}
	\item Usuário desconectado;
	\item Problemas com resposta do servidor ao pesquisar uma pessoa/grupo;
	\item Exibir a mensagem de erro ao usuário;
	\item Usuário pesquisa um nome que não existe na rede social;
	\item Exibir mensagem de pessoa/grupo não encontrado;
	\item Usuário não estar conectado a Internet.
\end{itemize}
}
%Frequência de Ocorrência
{Toda vez que o usuário requirir uma pesquisa de pessoa/grupo.}
%Problemas em aberto
{

}
%--------------------------------------------------------------------------------------------------------
\casoDeUso
%Identificador
{UC10}
%Nome
{Enviar mensagem no chat}
%Ator Principal
{Usuário}
%Interessados
{
\begin{itemize}
	\item Usuário: Enviar mensagem no \textit{chat} para outros usuários.
	
\end{itemize}

}
%Pré-Condições
 {Possuir uma conta válida e estar conectado (logado no sistema).}
%Pós-condições
{}
%Fluxo Básico
{
\begin{itemize}
	\item Usuário acessa o \textit{link} mensagens;
	\item O sistema mostra amigos \textit{online} ou \textit{offline} (Figura \ref{figura:chat_ativos_e_inativos});
	\item Usuário seleciona um amigo \textit{online};
	\item Usuário digita uma mensagem e clica no botão ``Enviar" (Figura \ref{figura:mensagens}).
\end{itemize}
}
%Fluxo Alternativo
{
\begin{itemize}
	\item Ocorreu falha na conexão com a Internet.
	\begin{itemize}
	\item O sistema apresenta uma mensagem para o usuário verificar os problemas com a conexão da Internet.
	\end{itemize}
\end{itemize}
}
%Frequência de Ocorrencia
{Diariamente.}
%Problemas em Aberto
{
}
%--------------------------------------------------------------------------------------------------------
\casoDeUso
%Identificador
{UC11}
%Nome
{Manter álbum}
%Ator Principal
{Usuário}
%Interessados
{
\begin{itemize}
	\item Usuário: seleciona a opção ``Fotos".
	
\end{itemize}

}
%Pré-Condições
 {Possuir uma conta válida e estar conectado (logado no sistema).}
%Pós-condições
{}
%Fluxo Básico
{
\begin{itemize}
	\item Usuário acessa o menu "Minhas fotos" (Figura \ref{figura:suas_fotos});
	\item O sistema redireciona para seus álbuns (Figura \ref{figura:seus_albuns});
	\item Usuário seleciona a opção que deseja: Criar álbum, Suas fotos, Seus álbuns ou as Fotos com você.
	
\end{itemize}
}
%Fluxo Alternativo
{
\begin{itemize}
	\item Ao selecionar a opção de "Suas fotos" conforme a (Figura \ref{figura:suas_fotos}). O usuário visualizará todas as fotos e ele pode: editar e deletar;
	\item Ao selecionar a opção "Seus álbuns" conforme a (Figura \ref{figura:seus_albuns}). O sistema mostra todos seus álbuns da rede social e o usuário pode: visualizar, editar e deletar as fotos do álbum;
	\item Ao seleciona a opção "Criar álbum", cria-se um novo álbum.
	
\end{itemize}
}
%Frequência de Ocorrencia
{Duas vezes por semana.}
%Problemas em Aberto
{

}
%--------------------------------------------------------------------------------------------------------
\casoDeUso
%Identificador
{UC12}
%Nome
{Apresentar Chat}
%Ator Principal
{Usuário}
%Interessados
{
\begin{itemize}
	\item Usuário: Deseja acessar o opção chat.	
\end{itemize}

}
%Pré-Condições
{Estar logado na rede social em sua conta.}
%Pós-condições
{}
%Fluxo Básico
{
\begin{itemize}
	\item Usuário seleciona a opção ``Mensagens", que pode ser observado através da (Figura \ref{figura:meus_posts});	
	\item O sistema abre o \textit{chat}, mostrando ao usuário seus amigos que estão \textit{online}  e \textit{offline}, igual a (Figura \ref{figura:chat_ativos_e_inativos});	
	\item O usuário seleciona qual contato deseja enviar mensagem;
	\item O sistema abre a janela para realizar o envio de mensagem como demonstrado no Caso de Uso: Enviar mensagem no chat (UC10).  
					
\end{itemize}
}
%Fluxo Alternativo
{
\begin{itemize}
	\item Ocorreu falha na conexão com a Internet ou servidor:
	\begin{itemize}
	\item O sistema apresenta uma mensagem de alerta pedindo para que o usuário verifique sua conexão com a Internet e aperte F5 para recarregar a página. 	
	\end{itemize}
	\item Usuário minimiza o \textit{chat}. 
\end{itemize}
}
%Frequência de Ocorrencia
{Aproximadamente 2 vezes por minuto.}
%Problemas em Aberto
{

}
%-----------------------------------------------------------
\casoDeUso
%Identificador
{UC13}
%Nome
{Solicitação de amizade}
%Ator Principal
{Usuário}
%Interessados
{
\begin{itemize}
	\item Usuário: Deseja solicitar amizade.
\end{itemize}

}
%Pré-Condições
{Ter uma conta válida e estar logado no perfil.}
%Pós-condições
{Amigo adicionado aceitar a solicitação.}
%Fluxo Básico
{
\begin{itemize}
		\item Usuário entra no perfil do amigo selecionado;
		\item Usuário clica no botão ``Adicionar aos amigos"" (Figura \ref{figura:solicitacao_amizade}).	
\end{itemize}
}
%Fluxo Alternativo
{
\begin{itemize}
	\item Ocorreu uma falha na conexão com a Internet:
	\begin{itemize}
		\item O sistema apresenta uma mensagem para o usuário verificar os problemas com a conexão da Internet.
	\end{itemize}
	
	\item Problemas com resposta do servidor ao adicionar o amigo:
	\begin{itemize}
	\item O sistema exibe uma mensagem informando que houve um problema com resposta do servidor.
	\end{itemize}

\end{itemize}
}
%Frequência de Ocorrencia
{Toda vez que o usuário requerir uma solicitação de amizade}
%Problemas em Aberto
{
 
}
%-----------------------------------------------------------
\casoDeUso
%Identificador
{UC14}
%Nome
{Aceite Solicitação de amizade}
%Ator Principal
{Usuário}
%Interessados
{
\begin{itemize}
	\item Usuário: Recebe uma Solicitação de amizade.
\end{itemize}

}
%Pré-Condições
{Amigo ter enviado uma solicitação de amizade.}
%Pós-condições
{Usuário aceitar a solicitação.}
%Fluxo Básico
{
\begin{itemize}
		\item Usuário irá receber uma notificação de amizade na(Figura X);
		\item Usuário clica no botão ``Aceitar", para aceitar o amigo.
\end{itemize}
}
%Fluxo Alternativo
{
\begin{itemize}
	\item Ocorreu uma falha na conexão com a Internet:
	\begin{itemize}
		\item O sistema apresenta uma mensagem para o usuário verificar os problemas com a conexão da Internet.
	\end{itemize}
	
	\item Problemas com resposta do servidor ao adicionar o amigo:
	\begin{itemize}
	\item O sistema exibe uma mensagem informando que houve um problema com resposta do servidor.
	\end{itemize}

\end{itemize}
}
%Frequência de Ocorrencia
{Toda vez que o usuário aceitar ou recusar uma solitação de amizade.}
%Problemas em Aberto
{
 
}
%-----------------------------------------------------------

