\chapter{Projeto}

Este capítulo tem como foco a apresentação ilustrativa do sistema por meio do Diagrama Entidade e Relacionamento(DER), bem como uma representação estrutural e dos relacionamentos das classes pelo Diagrama de Classes. Além disso, busca demonstrar as trocas de informações entre operações pelos Diagramas de Sequência e uma visão estática da estrutura física sobre a qual o \textit{software} será implementado pelo Diagrama de Implantação.

\section{Diagrama Entidade e Relacionamento (DER)}

O Diagrama Entidade e Relacionamento (DER) apresenta de forma gráfica as tabelas do banco de dados, mostrando os tipos de dados que estarão armazenados, bem como os relacionamentos existentes entre as tabelas.

A Figura \ref{figura:DER} mostra as tabelas do banco de dados, onde um usuário pode morar em uma cidade, que possui um estado, e este um país, o qual pode ter vários estados, que conseguinte pode ter várias cidades e logo, uma cidade vários usuários.

Um usuário pode ter vários álbuns, que pode ter várias multimídias. Bem como também pode ter várias postagens, que pode ter vários comentários e estes podem ter comentários também.

Uma postagem pode ter várias multimídias e várias multimídias podem estar em várias postagens, como também uma postagem pode ter vários álbuns e vários álbuns podem estar em várias postagens.

\figura{DER}{15}{../banco_de_dados/redesocial.png}{Diagrama Entidade e Relacionamento (DER).}

\newpage

\section{Diagrama de Classes}

Mostra um conjunto de classes e seus relacionamentos. É o diagrama central da modelagem orientada a objetos. Os elementos de um Diagrama de Classes são:
Relacionamento; Associação; Agregação; Composição; Generalização; Dependência. Como na Figura\ref{figura:multimidia_album_classdiagram} as classes são representadas por retângulos incluindo: nome, atributos e métodos.

\figura{multimidia_album_classdiagram}{15}{../documentacao/capitulos/projeto/diagramas/multimidia_album_classdiagram.png}{Diagrama de Classes Entidade Multimidia e Album.}

\figuraH{Diagrama_de_Classes_DTO}{15}{../projeto/Diagrama_de_Classes_DTO.png}{Diagrama de Classe DTO}

\figuraH{Diagrama_de_Classes_DAO}{15}{../projeto/Diagrama_de_Classes_DAO.png}{Diagrama de Classe DAO}

\figuraH{Diagrama_de_Classes_BO}{15}{../projeto/Diagrama_de_Classes_BO.png}{Diagrama de Classe BO}


\newpage
\section{Diagramas Casos de de Uso}
\figuraH{UCD_Geral}{15}{../projeto/UCD_Geral.png}{Diagrama Casos de Uso}

\section{Diagramas de Sequência}

\section{Diagrama de Implantação}
