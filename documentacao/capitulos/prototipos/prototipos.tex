\chapter{Protótipos}

A Figura \ref{figura:login} apresenta a tela de login, onde o usuário poderá logar, com seu usuário e senha, ou criar uma nova conta, preenchendo seus dados pessoais.

\figura{login}{15}{../prototipos/resultados/login.png}{Protótipo de login.}

A Figura \ref{figura:cadastro_usuario_parte2} apresenta a tela onde o usuário deverá confirmar seu e-mail, ou altera-ló caso tenha digitado errado, ou queira mudar de endereço eletrônico.

\figura{cadastro_usuario_parte2}{13}{../prototipos/resultados/cadastro_usuario_parte2.png}{Protótipo da segunda etapa de cadastro, confirmação do e-mail.}

A Figura \ref{figura:cadastro_usuario_parte3} apresenta a tela que o usuário preencherá com dados pessoais sobre seu perfil e informações da sua vida aos quais serão disponibilizadas para os outros usuários que tenham acesso.

\figura{cadastro_usuario_parte3}{16}{../prototipos/resultados/cadastro_usuario_parte3.png}{Protótipo da terceira etapa de cadastro, informações pessoais.}

A Figura \ref{figura:cadastro_usuario_parte4} apresenta a tela onde o usúario adicionará foto ao seu perfil, nesta terá a opção de escolher uma foto que esteja em seus arquivos no computador ou tirar uma instantaneamente pela \textit{WebCam}.

\figura{cadastro_usuario_parte4}{16}{../prototipos/resultados/cadastro_usuario_parte4.png}{Protótipo da quarta etapa de cadastro e foto de perfil.}

A Figura \ref{figura:perfil} apresenta o perfil inicial da rede social, contendo todas as informações, fotos e publicações do usuário.

\figura{perfil}{15}{../prototipos/resultados/perfil.png}{Protótipo de perfil.}

A Figura \ref{figura:postagem} apresenta a tela de postagem, que poderá ser texto, foto, vídeo, sentimentos e localização, além de poder marcar amigos.

\figura{postagem}{15}{../prototipos/resultados/postagem.png}{Protótipo de postagem.}

A Figura \ref{figura:meus_posts} apresenta a tela com todas as suas postagem, o usuário também poderá fazer novas publicações.

\figura{meus_posts}{19}{../prototipos/resultados/meus_posts.png}{Protótipo de posts pessoais.}

A Figura \ref{figura:chat_ativos_e_inativos} apresenta a tela com todas os amigos de um usuário, que esteja ativos ou inativos.

\figura{chat_ativos_e_inativos}{10}{../prototipos/resultados/chat_ativos_e_inativos.png}{Protótipo de chat de amigos ativos e inativos.}

A Figura \ref{figura:mensagens} apresenta a tela de chat de mensagens, o usuário poderá conversar com seus amigos, mandando mensagens de texto.

\figura{mensagens}{10}{../prototipos/resultados/mensagens.png}{Protótipo de chat de mensagens.}

A Figura \ref{figura:recuperacao_senha} permite que os usuários recuperem suas senhas, digitando o seus respectivos e-mails.

\figura{recuperacao_senha}{15}{../prototipos/resultados/prototipo_recuperacao_senha.png}{Protótipo de recuperação de senha.}

A Figura \ref{figura:galeria_fotos} exibe todas as fotos do usuário, separadas por categorias  e da a opção de criar novos álbuns de fotos.

\figura{galeria_fotos}{19}{../prototipos/resultados/galeria_fotos.png}{Protótipo da galeria de fotos.}

A Figura \ref{figura:solicitacao_amizade} permite que o usuário convide novos amigos.

\figura{solicitacao_amizade}{10}{../prototipos/resultados/solicitacao_amizade.png}{Protótipo de solicitação de amizade.}

A Figura \ref{figura:entrar_grupo} exibe a descrição do grupo e a opção de entrar nesse grupo.

\figura{entrar_grupo}{15}{../prototipos/resultados/entrar_grupo.png}{Protótipo para entrar em grupo.}

A Figura \ref{figura:criar_grupo} permite criar um novo grupo, preenchendo um nome ao grupo, convidando amigos, definindo o tipo do grupo e uma imagem.

\figura{criar_grupo}{15}{../prototipos/resultados/criar_grupo.png}{Protótipo de criação de grupo.}

A Figura \ref{figura:denunciar_grupo} apresenta a tela de denúncia de grupo, onde o usuário informa o motivo da denúncia.

\figura{denunciar_grupo}{15}{../prototipos/resultados/denunciar_grupo.png}{Protótipo denúncia de grupo.}

A Figura \ref{figura:opcoes_grupo} apresenta a tela de opções do grupo, onde se tem as opções disponíveis do grupo.

\figura{opcoes_grupo}{15}{../prototipos/resultados/opcoes_grupo.png}{Protótipo de opções de grupo.}

A Figura \ref{figura:cadastro_artigos} apresenta a tela de cadastro de artigos, que deve ser preenchida.

\figura{cadastro_artigos}{15}{../prototipos/resultados/cadastro_artigos.png}{Protótipo de cadastro de artigos.}

A Figura \ref{figura:sair_grupo} apresenta a tela de confirmação para sair do grupo.

\figura{sair_grupo}{15}{../prototipos/resultados/sair_grupo.png}{Protótipo de sair do grupo.}

A Figura \ref{figura:painel_admin} apresenta uma tela com visão geral das informações da rede social.

\figura{painel_admin}{15}{../prototipos/resultados/painel_final.png}{Protótipo de painel administrativo.}

A Figura \ref{figura:seus_albuns} apresenta a tela com todos os álbuns de um usuário.

\figura{seus_albuns}{15}{../prototipos/resultados/Seus_Albuns.png}{Protótipo de seus álbuns.}

A Figura \ref{figura:suas_fotos} apresenta a tela com todas as fotos de um usuário.

\figura{suas_fotos}{15}{../prototipos/resultados/Suas_fotos.png}{Protótipo de suas fotos.}

